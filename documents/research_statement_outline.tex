\documentclass{article}
\usepackage[margin=0.5in]{geometry}
\begin{document}
\section{Introduction}
\begin{enumerate}
\item I am currently a PhD student at the University of Illinois at Urbana-Champaign working with professor Lucas Wagner focusing on \textit{ab-initio} simulation and model downfolding for condensed matter systems.
\end{enumerate}

\section{Undergraduate research}
\begin{enumerate}
\item During my undergraduate studies at the University of Illinois at Urbana-Champaign, I worked with professor Karin Dahmen investigating the universality of quake statistics in soft-condensed matter systems.
\end{enumerate}

\section{PhD research}
\begin{enumerate}
\item My PhD research was initially focused on generating accurate low-energy states for condensed matter systems  using \textit{ab-initio} Quantum Monte Carlo (QMC).

\item In my first project during my PhD research, I investigated compact representations of ground state wave functions in QMC.

\item I followed up this starter project with a methodological work in efficiently computing high-variance quantities required in QMC ground state wave function optimization.

\item In my third project, I worked on developing and testing a QMC method which extends previous ground state optimization techniques to a penalty-based excited state optimization technique.

\item The last two projects I have been working have shifted focus to developing effective  Moire-scale models for the novel twisted-bilayer graphene system (TBLG) as part of a large collaborative effort called QMC-HAMM. 

\item The first of these projects has been focused on developing an atomic-scale lattice-deformation dependent tight-binding model for single- and bi-layer graphene systems.

\item The latter project is oriented around developing an accurate atomic-scale effective model for single-layer graphene with long-range interactions.
\end{enumerate}

\section{Future research plan (if applying for generic)}
\begin{enumerate}
\item My short term research plan is focused on finishing up an accurate Moire-scale model for twisted-bilayer graphene, a task which would take three big steps.

\item The first step, developing an atomic-scale lattice-deformation interacting model for single-layer graphene, would key us into the interesting effects of lattice-electronic interactions in graphene based systems.

\item The second step, developing atomic-scale interacting models for simple AB, AA and SP stacked bi-layer graphene systems would begin to explain the important variations in interactions in twisted-bilayer systems.

\item The third step is a culmination of the effort, leading to an atomic-scale deformation-dependent interacting model for twisted bilayer graphene.

\item In the longer term, I am particularly interested in model development for strongly interacting systems using \textit{ab-initio} computational techniques, and envision a two-pronged approach towards building a robust technique for model downfolding. 
The first prong will be oriented around developing methodology and frameworks for downfolding which are easily usable, and demonstrating the ability of these frameworks in developing interacting models for well-understood systems.

\item The second prong is more boundary pushing, and would involve pushing the methods and frameworks beyond their basic usage to develop models for novel, poorly-understood, strongly-interacting materials.
\end{enumerate} 
\end{document}
